\documentclass[a4paper,9pt]{ctexart}


%\usepackage{fancyhdr}
\usepackage{amsmath,amssymb}
\usepackage{graphicx}
%\usepackage[hmargin=1.25in,vmargin=1in]{geometry}
\usepackage{pdfpages}
\usepackage[colorlinks,
            linkcolor=red,
		 urlcolor=purple]{hyperref}
\usepackage{cleveref}
\usepackage{float}

\crefname{equation}{}{}
\crefname{figure}{图}{图}
\crefname{footnote}{注释}{注释}
\crefname{table}{表}{表}

%\cpic{<尺寸>}{<文件名>}}用于生成居中的图片。
\newcommand{\cpic}[2]{
\begin{center}
\includegraphics[width=#1\textwidth]{#2}
\end{center}
}

%\cpicn{<尺寸>}{<文件名>}{<注释>}用于生成居中且带有注释的图片,其label为图片名。
\newcommand{\cpicn}[3]
{
\begin{figure}[H]
\cpic{#1}{#2}
\caption{#3\label{#2}}
\end{figure}
}

\newcommand{\beq}{\begin{equation}}
\newcommand{\eeq}{\end{equation}}
\newcommand{\bea}{\begin{equation}\begin{aligned}}
\newcommand{\eea}{\end{aligned}\end{equation}}

%输入单位和数学常数
%下面所有命令需在公式环境下使用
\newcommand{\e}{\mathrm{e}}   %自然常数e = \e
\newcommand{\im}{\mathrm{i}}   %虚数单位i = \im
\newcommand{\meter}{\mathrm{m}}      %单位/前缀 = \单位/前缀英文名
\newcommand{\newton}{\mathrm{N}}  
\newcommand{\joule}{\mathrm{J}}
\newcommand{\second}{\mathrm{s}}
\newcommand{\gram}{\mathrm{g}}
\newcommand{\ampere}{\mathrm{A}}
\newcommand{\kilo}{\mathrm{k}}
\newcommand{\milli}{\mathrm{m}}
\newcommand{\kelvin}{\mathrm{K}}
\newcommand{\mole}{\mathrm{mol}}
\newcommand{\volt}{\mathrm{V}}
\newcommand{\nano}{\mathrm{n}}
\newcommand{\degreeC}{^\circ \mathrm{C}}  %摄氏度符号 = \degreeC

\newcommand{\so}{ $\Rightarrow$ }
\newcommand{\info}[1]{{\bf \color{red} #1}}

\newcommand{\reals}{\mathbb{R}}
\newcommand{\complexs}{\mathbb{C}}
\newcommand{\ints}{\mathbb{Z}}
%\newcommand{\dim}{\mathrm{dim\ }}
\newcommand{\up}{\uparrow}
\newcommand{\down}{\downarrow}
\newcommand{\del}{\vec \nabla}
\newcommand{\su}{\mathfrak{su}}
\newcommand{\lbk}{\left(}
\newcommand{\rbk}{\right)}

\DeclareMathOperator{\tr}{tr}
\DeclareMathOperator{\diag}{diag}
\newcommand{\card}{\mathrm{card \ }}
\newcommand{\mani}{\mathcal{M}}
\newcommand{\lag}{\mathcal{L}}
\newcommand{\ham}{\mathcal{H}}
\def\secpage#1#2{\begin{frame}\bch\bcenter{\bf \Huge #1} \skipline \tbox{#2}\ecenter\ech\end{frame}}
\newcommand{\mat}[1]{\begin{pmatrix}#1\end{pmatrix}}
\newcommand{\mev}{\ {\rm MeV}}
\newcommand{\pfrac}[2]{\frac{\partial #1}{\partial #2}}

\newcommand{\unit}[1]{{\rm \ #1}}
\newcommand{\emptyline}{\par \ \\ }

\newtheorem{eg}{例}[section]
\newtheorem{ans}{解答}[section]


\def\bropt{\,(\ \ \ )}
\def\optlist#1#2#3{(A)\;#1\ \ \ (B)\;#2\ \ \ (C)\;#3}
\def\ooptlist#1#2#3#4{(A)\;#1\ \ \, (B)\;#2\ \, \ (C)\;#3\ \, (D)\;#4}


\title{高中物理讲义}
\author{高寒}
\begin{document}
\maketitle
\tableofcontents
\newpage
\section{引言、质点和运动的描述}
\subsection{高中物理的研究对象}
\emph{目前来看,你觉得高中物理都讲了些什么?研究什么问题?未来可能继续学习什么内容?}
\begin{itemize}
\item
如果要理一条主线,高中物理主要是讲\info{质点}在各种\info{力}的作用下如何\info{运动},也就是质点牛顿第二定律的应用。
\item
全国卷高考物理有2道大题,一般都围绕这个话题展开。一般一道题涉及电、另一题为纯力学,且喜欢考直线运动(e.g. 木板滑块模型) \so 高一上的知识点是整个高中物理的基础,也是高考的重点。
\end{itemize}
\emph{怎么学习?}
\begin{itemize}
\item
多阅读教材,熟悉基本\info{概念}:了解概念背后的\info{数学工具},熟悉定律定理对应的典型物理\info{现象};
\item
多想一些实际例子,建立清晰的物理图像 \so 对题目能够很清晰地想清楚物理过程,这对高考的物理综合大题非常重要!
\end{itemize}
\subsection{质点}
\emph{怎么理解质点?}
\begin{itemize}
\item
不妨把我们的问题放大一点,看看怎么理解物理中的各种模型。
\item
质点可以理解成数学中的点,它的唯一属性是质量$m$(未来还会引入电荷$q$)。\so 高度抽象化和理想化。
\item
如果题目没有说,质点\info{不会碎};在高中阶段,所有的质点都\info{不会转}。
\begin{itemize}
\item
不会碎:质点永远只是一个点,就像小时候玩的弹珠:只要不用太大的力,它永远都是一个坚硬的球体。(著名段子:坚不可摧的小滑块)
\item
不会转:质点不会自己转动。高中阶段不考虑转动。下面的问题具有启发性:
\end{itemize}
\end{itemize}
\begin{eg}
一个玻璃弹珠从倾角为$\theta$的斜面下滚下,其加速度大小是多少?是$g\sin \theta$吗?
\end{eg}
\begin{ans}
答案不是著名的$g\sin \theta$。因为弹珠是从斜面上滚下而不是滑下,它会越滚越快从而消耗一部分重力势能,使得加速度小于$g\sin \theta$。真实的答案是$a=\frac{5}{7}g\sin \theta$。(怎么得到这个答案的?超纲啦,别管)
\end{ans}
如果考虑的是小滑块,那么加速度就是$g\sin \theta$。 \so 高中物理不考虑质点的转动!
\begin{itemize}
\item
因此,老生常谈的例题:研究地球绕太阳转的时候,可以把地球当初质点;研究地球自转的时候就不行。
\end{itemize}
物理中的其他模型也是这样高度抽象化的,当然,考虑清楚什么时候这种抽象是合理的是一种重要的,需要慢慢培养的能力。来看另外一个抽象的例子:
\begin{eg}
将一个质量为$2\unit{kg}$的物块靠近一个粗糙的竖直墙面后松手,请问墙面和滑块间有没有\info{摩擦力}的作用?墙面和滑块间有没有\info{力}的作用?
\end{eg}
\begin{ans}
滑块对墙面没有正压力的作用 \so 由$f = \mu N$,不存在摩擦力作用。
\par
但有没有\info{力}的作用呢?肯定是没有了。但是比较杠的同学可能会说,任何两个物体都存在万有引力,为什么不考虑呢?
\par
两个物体间万有引力的公式为$F = \frac{Gm_1m_2}{r^2}$,$G$在国际单位制下大概是$6.7\times 10^{-11}$,两个相隔$0.1\unit{m}$,质量在kg量级的物体间万有引力只有$\sim 10^{-9} \unit N$。而这里我们考虑的力都在$mg \sim 10\unit{N}$的量级,因此在这种问题里我们就当万有引力不存在了。
\end{ans}
同样的哲学:为什么研究地球公转的时候不用考虑自转?地球半径$R_\oplus \approx 6000\unit{km}$;地日距离$r \approx 2\times 10^8 \unit{km}$,相比起来实在太小了!
\subsection{运动的描述}
\emph{我们已经同意了质点就是一个数学上的点,那么,怎么描述运动?}
\begin{itemize}
\item
数学和物理是不可分离的整体,在学习物理时要主动借用初高中数学中学到的工具!即使高中物理老师常常回避这个问题 \so 初中的平面直角坐标系、三角函数;高中将要学到的向量和导数都是帮助理解物理概念的重要数学工具,在将来学到这些数学时要主动跟物理联系起来。
\item
数学上,点的位置用坐标系描述,点的坐标为$(x,y)$。高一上学期,我们考虑的问题更简单:质点只在$x$轴上运动 \so 我们只需要一个坐标$x$。
\item
质点的坐标$x$是时间$t$的函数,就是一个类似于抽象成$f(x)$的关系式$x(t)$ \so 如果$x(t)$的表达式已知,质点的运动也就清楚了。
\item
数学里函数图像是研究函数的重要工具 \so 我们也把$x(t)$做成一个图,并且物理老师给它起了一个有点可怕的名字,叫$x-t$图。
\end{itemize}
\begin{eg}
以$2\unit{m/s}$的速度匀速运动的质点在$t=3\unit{s}$时位于$x=3\unit{m}$的位置,它的位置时间关系写成公式是什么样子?画成$x-t$图是什么样子呢?
\end{eg}
\begin{ans}
虽然这是一个很简单的问题,但是请在下面自己写写吧:
\vspace{5cm}
\end{ans}



\subsection{速度和加速度}
\emph{为什么匀速直线运动的平均速度就等于瞬时速度?}
\begin{itemize}
\item
质点在$(t,t+\Delta t)$时间段里的\info{平均速度}定义为
\beq
\bar v = \frac{x(t+\Delta t) - x(t)}{\Delta t}
\eeq
而质点在$t$时刻的\info{瞬时速度}就是令$\Delta t \to 0$的$\bar v$。
\begin{itemize}
\item
平均速度对一个时间段定义,而瞬时速度对一个时间点定义。
\item
令$\Delta t \to 0$得到瞬时速度的操作也可以对任意的数学函数$f(x)$做,它在数学上叫做求导数,我们会在高二下学期的数学课上学习到这种操作和导数的具体计算,从而更好地回来理解这一块知识点。不过我们可以看一个简单的例子体验一下:
\begin{eg}
如果一个质点的运动规律是$x = t^2 + 3$,求它在$(2\unit{s},2\unit s + \Delta t)$时间段内的平均速度和$2\unit s$时的瞬时速度。
\end{eg}
\begin{ans}
按定义
\beq
\bar v = \frac{(2+ \Delta t)^2 + 3 - (2^2 + 3)}{\Delta t} = \frac{4 + 4\Delta t + \Delta t^2 - 4}{\Delta t} = 4 + \Delta t
\eeq
从而再让$\Delta t$趋于零得到$t=2\unit s$的瞬时速度$v = 4\unit{m/s}$。
\end{ans}
\item
对应在$x-t$图上,在$t$时刻质点的瞬时速度就是$x-t$曲线在那一点处的斜率(画个图在下面解释一下).
\vspace{5cm}
\end{itemize}
现在你应该能回答我们开头提出的问题了!
\vspace{5cm}
\item
把上面的所有$x$改成$v$;$v$改成$a$就是平均加速度和瞬时加速度的定义。
\begin{itemize}
\item
速度$v$衡量位置$x$的变换快慢;而$a$就相应地衡量$v$的变换快慢。
\item
就像某时刻的速度$v$和位置$x$毫无关系一样,某时刻的$a$也和$v$毫无关系(e.g. 刚点火的火箭具有巨大的$a$,而$v$几乎为0;在高空巡航的客机有很大的$v$,而$a$几乎为0)。
\item
学完高一上学期应该很清楚了:为什么研究加速度?因为它出现在我们的中心定律$F=ma$的右端!
\item
对$x = \frac{1}{2}at^2 + v_0 t$做一次“求导数”操作得到$v(t)$函数;再对$v(t)$函数做一次“求导数”操作可以得到加速度$a$,从而可以自己验证这个公式的正确性。感兴趣就自己动手试试吧!
\end{itemize}
\item
研究这种问题的一个更加powerful的工具就是$v-t$图,它就是函数关系$v(t)$的函数图像。
\begin{itemize}
\item
中心结论:$v-t$曲线和$t$轴围成的面积是质点在这段时间内的位移$\Delta x$;$v-t$曲线在某点处的斜率是质点在这个时刻的速度。
\item
上面两个东西都区分正负:如果曲线在$t$轴下方,则围成面积为负,位移也为负(为什么?曲线在$t$轴下方 \so $v<0$ \so 质点朝$-x$方向运动);如果斜率为负,加速度$a$就为负(为什么?斜率为负表示速度在减小)。
\item
斜率为0的点,质点速度达到最大或最小值。
\item
看\cref{vtdiagram}...
\cpicn{0.8}{vtdiagram}{一个典型的$v-t$图}
\item
$v-t$曲线为倾斜直线 \so 斜率为常数 \so $a$为常数 \so 质点做匀加速直线运动 ( 曲线就一定是变加速!)
\end{itemize}
\end{itemize}
下面一个题有一定难度,不过如果你能理清这例题,你就能理清$v-t$图这一节的所有概念!
\begin{eg}
质点的运动规律是$x = 2\sin  t\  ({\rm m})$,如\cref{sinecurve},试大概画出对应的$v-t$图,并指出质点的加速度什么时候达到最大?质点的位置什么时候达到最大?对于这个运动你还能说出一些什么特征?
\cpicn{0.6}{sinecurve}{$x = 2\sin  t ({\rm m})$对应的$x-t$图}
\end{eg}
\begin{ans}
自己动手做一做!
\vspace{8cm}
\end{ans}





\newpage
\section{匀加速直线运动的规律}
我们上一讲的末尾介绍了高中物理大杀器$v-t$图。这一节我们来小试牛刀,用它来复习一下匀加速直线运动。
\subsection{速度时间关系}
\emph{匀加速直线运动的规律那么多条,怎么记住?}
\begin{itemize}
\item
答案是,你只用记一条:
\beq
v = at + v_0
\eeq
\emph{说一说上面这个公式里每个参数的物理意义?}
\end{itemize}
\subsection{位移时间关系和位移速度关系}
\emph{后面那么多怎么办呢?}
\begin{itemize}
\item
先画出$v = at + v_0$对应的$v-t$图。利用规律“$v-t$图和$t$轴围成的面积就是质点的位移“,得到
\beq
x = \frac{1}{2}at^2 + v_0 t
\eeq
\item
怎么用$v-t$图推出速度位移关系?看\cref{vxrel}。
\cpicn{0.8}{vxrel}{用$v-t$图得到速度位移关系}
\emph{提示够多啦,请自己写出完整的推导过程。}
\vspace{4cm}
\end{itemize}
\begin{eg}
解释公式$\Delta s = aT^2$的中各个符号的物理含义,在一个$v-t$图上画出各个物理量并用$v-t$图证明之。
\end{eg}
\begin{ans}
自己动手做一做!
\vspace{6cm}
\end{ans}

\emph{老师好无聊,为什么非要用$v-t$图这种东西研究?}\par
可视:容易建立清晰的物理图像。在力学大题中一般涉及多个过程\so $v-t$图可以更方便地研究相对运动趋势;判断摩擦力方向等,因此最好从最基本的物理开始就熟悉运用$v-t$图这种工具!
\subsection{追击问题}
\begin{itemize}
\item
追击问题的最有效解决工具就是$x = \frac{1}{2}at^2 + v_0t + x_0$,然后就可以转换成初中二次函数求最值问题无脑做了。
\item
不过也要理清楚物理图像,比如问一下自己,为什么两者速度相等时相距最大/最小?和你直接用二次函数求最值的方法一致吗?
\end{itemize}
\begin{eg}
静止的警车发现前方$100\unit m$处有一辆违法车辆以$72\unit{km/h}$的速度行驶,警车立即以$a = 2\unit{m/s^2}$的恒定加速度追赶。由于当天大雾,当两车相距$240\unit{m}$后便会失去视野而无法再跟踪车辆。请问警车能否追上违法车辆?
\end{eg}
\begin{ans}
以警车为原点建立一维坐标系$Ox$,警车的速度-时间关系写为
\beq
x_1 = t^2
\eeq
犯罪车辆的速度-时间关系为
\beq
x_2 = 20t + 100
\eeq
两者的距离差为
\beq
s = x_2 - x_1 = -t^2 + 20t + 100
\eeq
根据数学中的二次函数知识,当$t = \frac{20}{2}\unit s = 10 \unit s$时,$s$有最大值$-10^2 + 200 + 100 = 200({\rm m})$,小于240 m。此后警车开始靠近违法车辆,故可以追上。
\end{ans}
\subsection{其他习题}
\begin{itemize}
\item
其他习题主要涉及匀加速直线运动的复杂计算和读$v-t$图。主要运用的工具还是$v-t$图,它能很好地帮你看清问题。
\item
灵活运用$v^2 - v^2_0 = 2ax$和$\Delta s = aT^2$有时可以很大地减小计算量。
\end{itemize}
\begin{eg}
坐标轴上有原点$O$和三点$ABC$,其中$AB = s_1, BC = s_2$。一静止质点在$t=0$时刻开始从原点做匀加速直线运动,通过$AB$段和通过$BC$段的时间相等,求$OA$长度。
\end{eg}
\begin{ans}
画出$v-t$图:
\vspace{4cm}
\\
假设相等的时间是$T$,则加速度为$a = $\hspace{1cm},且$B$点速度为$v_B = $\hspace{1cm},最后在$v-t$图上补完所有可以推得的物理量求得$OA=$\hspace{3cm}。
\end{ans}





\newpage

\section{静力平衡}
\subsection{矢量合成和正交分解}
\emph{什么是矢量?力和速度是矢量吗?电流是矢量吗?}
\begin{itemize}
\item
矢量有大小和方向,还必须符合平行四边形定则(或三角形定则,更常用)。
\item
数学上的向量:在坐标系中表示成一个数组$\vec a = (a_x,a_y)$ \so 物理上的jargon:矢量的正交分解(i.e. 正交=两个垂直方向)。
\item
常用的正交分解方案:遇到斜面沿斜面;其他问题沿竖直和水平方向。
\item
合力和分力:等效替代的思想。
\item
平衡条件:系统内各个质点受到的合外力在$x,y$两个方向上都为0
\bea
F_{1x} + F_{2x} + \dots F_{nx} &= 0 \\
F_{1y} + F_{2y} + \dots F_{ny} &= 0
\eea
区分方向:正交分解后,沿$x$正方向的力写正号;负方向写负号(和速度一样)。
\end{itemize}
\begin{eg}
若质点在$n$个力的作用下平衡,则力$F_1,\dots, F_{n-1}$的合力$F_t$和$F_n$的关系是什么?
\end{eg}
\begin{ans}
自己试一试!如果一下想不出来,先考虑$n=2,3$的情况...
\vspace{2cm}
\end{ans}
\begin{itemize}
\item
受力分析:整体法与隔离法。结合生活实际经验,先定性分析想清楚图像!顺序:重力$\to$弹力$\to$摩擦力(为什么有这个顺序?)
\item
\emph{整体法的物理依据是什么?}
\end{itemize}
\subsection{常见力的特征}
\emph{杆和绳有什么区别?}
\begin{itemize}
\item
重力:竖直向下,正比于质量(生活常识)
\beq
G = mg
\eeq
质量和重量的区别:日常生活中不区分,物理上?\so 质量用$m = F/a$定义;重量是在地球表面测到的$mg$值 \so 在月球表面,因为月球的$g$大概是地球的$1/6$,同样的物体重量也变成原来的$1/6$,但是质量不变。
\item
弹力:与形变方向相反,大小正比于形变量:$F = kx$。\so \info{硬}杆是什么意思?$k \to \infty$使得形变可以忽略不计,提供任意的弹力。
\item
摩擦力
\begin{itemize}
\item
不管是静摩擦力还是滑动摩擦力,总是阻碍相对运动(趋势):不准离开我
\item
静摩擦力:有一个最大值,一般题目不明说可以认为近似等于滑动摩擦力$f = \mu N$。 \so 如果没有沿表面的压力$N$,任何情况下也不会有静摩擦力。
\item
滑动摩擦力:$f = \mu N$,只取决于这两个因素!
\end{itemize}
高二开始会学习简单的电场力和磁场力,不过也万变不离其中。
\end{itemize}
\subsection{典型习题}
\begin{itemize}
\item
斜面滑块问题:
\end{itemize}
\begin{eg}
固定斜面上放置一个物块,认为最大静摩擦力等于滑动摩擦力,摩擦系数为$\mu$,倾角为$\alpha$。如果物块能够沿斜面滑下,$\mu$和$\alpha$之间应该满足?
\end{eg}
\begin{ans}
思路:定性分析:生活经验,$\theta$越大,$\mu$越小,越容易滑下。\par
做法:正交分解,最大滑动摩擦力等于\hspace{4cm},
\vspace{5cm}
\end{ans}
\begin{itemize}
\item
自锁问题:
\end{itemize}
\begin{eg}
拖把和地面的摩擦系数为$\mu$,拖把自重不计。拖把杆与地面夹角至少为多大时,无论使多大的力都无法推动拖把?
\end{eg}
\begin{ans}
思路:定性分析:生活经验,$\theta = 90^\circ$时,肯定没法推动;$\theta = 0^\circ$时,一点点力就可以推动(为什么?)\so $\theta$越大,越难推动。\par
做法:正交分解,最大滑动摩擦力大于\hspace{4cm},人施加的水平方向分力(动力)等于\hspace{4cm}。要推动,动力要大于\hspace{4cm}。从而列出等式...
\vspace{5cm}
\end{ans}
\emph{如果考虑拖把自重呢?刚才的定性分析不会有太大变化,具体计算?试一试!(如何快速检查你的结果?和上题比较,取极限...}
\vspace{7cm}
\begin{itemize}
\item
摩擦力叠叠乐:先整体再隔离的一般思路。
\end{itemize}

\newpage
\section{牛顿第二定律}
\subsection{牛顿第二定律的理解}
\begin{itemize}
\item
牛顿第二定律是连接运动学和力学的桥梁
\beq
F = ma
\eeq
\item
定量地定义了惯性(惯性=质量):质量是物体被外力改变运动状态的难易程度。
\item
矢量等式:$\vec F = m\vec a$,在平面内找两个正交方向$x,y$,实际上牛顿第二定律是两个方程:
\bea
F_x &= ma_x \\
F_y &= ma_y
\eea
\item
牛顿第二定律是一个\info{方程},也就是等号两边描述的是不同的对象。 \so 绝不能把加速度和力混为一谈。 \so 运用牛顿第二定律解题时,\info{坚持左边写合外力、右边写加速度的原则!}。
\end{itemize}

\subsection{整体法与隔离法}
\begin{itemize}
\item
牛顿第二定律具有\info{可加性}:假设有两个物体$A,B$,分别受到外力$\vec F_A,\vec F_B$,加速度为$\vec a_A,\vec a_B$,且存在相互作用力:$A$对$B$为$\vec F_{AB}$,$B$对$A$为$\vec F_{BA}$。 \so 分别列出各自的牛顿第二定律
\bea
\vec F_A + \vec F_{BA} &= m_A \vec a_A \\
\vec F_B + \vec F_{AB} &= m_B \vec a_B
\eea
\so 两式相加,牛顿第三定律保证了$\vec F_{AB} + \vec F_{BA} = 0$ \so 
\beq
\vec F_A + \vec F_B = m_A \vec a_A + m_B \vec a_B
\eeq
\item
物理诠释:将$A,B$打包成一个整体系统,系统受的总合外力等于系统内部的所有$m\vec a$相加。 \so \info{整体法}的物理依据。
\end{itemize}
\begin{eg}
两个用弹簧连接的物块在空中自由下落,上方物块的质量为$m_1 = 2\unit{kg}$,下方物块的质量为$m_2 = 1\unit{kg}$,某一时刻测得上方物块向下掉落的加速度为$8\unit{m/s^2}$,求下方物块此时的加速度和弹簧的弹力。指明弹簧此时是压缩还是拉伸。
\end{eg}
\begin{ans}
先整体法:打包成整体,只受重力。\so 列出等式
\beq
(m_1+m_2) g = m_1 a_1 + m_2 a_2
\eeq
\so $a_2 = $\hspace{2cm}。
\par
再对随意一个物块单独分析:请自己完成这部分。
\vspace{3cm}
\end{ans}

\begin{eg}
一个倾角为$\theta$的粗糙斜面固定在水平面上,其上放置一个质量为$m$的物块,物块以$a$的加速度下滑,地面对斜面的支持力是多大?朝什么方向?地面对斜面的摩擦力力是多大?朝什么方向?
\end{eg}
\begin{ans}
正交分解后在水平和竖直方向上分别列整体法的等式:
\vspace{6cm}
\end{ans}

\subsection{量纲分析}
\emph{为什么初中教$g = 9.8\unit{N/kg}$,高中教$g = 9.8\unit{m/s^2}$?}
\begin{itemize}
\item
物理的等式不仅是数的等式,更是量的等式 \so 带单位计算,等式左右两边必须是同类单位
\end{itemize}
\begin{eg}
判断以下表达式是否可能正确。
\begin{enumerate}
\item
$s = v^2/t$
\item
$v = s_1 s_2/t$
\end{enumerate}
\end{eg}
\begin{ans}
第一个等式左边单位为${\rm m}$,右边为${\rm m^2/s^3}$,非同类单位,不可能正确。
第二个
\end{ans}
\begin{itemize}
\item
牛顿第二定律给出了力的单位:$1 \unit N = 1\unit{kg\cdot m/s^2}$。
\item
量纲分析:高中力学只有3个基本单位${\rm m,s,kg}$,其他单位,如牛顿${\rm N}$,焦耳${\rm J}$都可以写成这三个基本单位的乘积。 \so 用单位来排除选择题错误答案,快速检查自己计算题结果。
\end{itemize}
\begin{eg}
单摆的周期(摆动一个往返的时间)$T$可能和摆长$l$,重力加速度$g$以及摆的质量$m$有关,下面哪个可能是$T$的正确公式?\\
\ooptlist{$T = 2\pi \frac{l}{g}$}{$T = 2\pi \frac{l}{\sqrt{mg}}$}{$T = 2\pi \frac{mg}{l}$}{$T = 2\pi \sqrt{\frac{l}{g}}$}
\end{eg}
\begin{ans}
自己试试,算出各选项右边表达式对应的单位。
\vspace{5cm}
\end{ans}



\newpage
\section{多过程的牛顿第二定律分析}
\emph{应用牛顿第二定律的关键是求出各物体在各阶段的加速度,再用$v-t$图分析运动。}
\subsection{单物体的多过程分析:斜面-平面模型}
\begin{itemize}
\item
从斜面上滑下再停止:对每个过程分别列出牛顿第二定律后算出每个过程的加速度,用$v-t$图分析。
\end{itemize}
\begin{eg} \label{egslope}
如\cref{slope},$m=0.5\unit{kg}$的质点从距斜面低端$L = 25\unit m$的地方滑下,斜面的倾角为$\theta = 37^\circ$,质点与斜面的动摩擦因数为$\mu_1 = 0.5$,与水平地面的动摩擦因数为$\mu_2 = 0.4$,求质点运动的总时间。
\end{eg}
\cpicn{0.6}{slope}{斜面-平面模型}
\begin{ans}
受力分析:在斜面上,质点受到重力$mg$、斜面对质点的支持力$N$和斜面对质点的摩擦力$f = \mu_1 N$,在垂直斜面方向列出牛顿第二定律
\beq
mg\cos \theta - N = 0
\eeq
在平行斜面方向列出牛顿第二定律
\beq
mg\sin \theta - f = ma_1
\eeq
从中解得$a_1 =$
\par
在水平面上,在竖直和水平方向分解,质点的加速度为$a_2 = \mu_2 g = 4\unit{m/s^2}$,方向向左。
\par
质点将在水平面上一直减速运动直到停止,画出过程的$v-t$图,算出关键点的坐标
\vspace{6cm}
\end{ans}
思考:\emph{给定$m = 0.5\unit{kg}$是多余条件吗?量纲分析的角度...}
\begin{itemize}
\item
如果水平面变成传送带,如何处理?摩擦减速过程结束=质点静止=质点和水平面速度相等(都为0)\so 在传送带上:摩擦减速过程结束=质点和传送带速度相等。(生活经验:安检传送带)
\end{itemize}
\begin{eg}
其他条件不变,将上面例题里的地面换成以$v_0 = 2\unit{m/s}$向右运行的传送带,其他条件不变,怎么解决?
\end{eg}
\begin{ans}
牛顿第二定律求加速度的部分不会变化,唯一变化的是...
\vspace{6cm}
\end{ans}
\begin{itemize}
\item
如果物体初速度不为零,需要分成几个过程分析?
\end{itemize}
\begin{eg}
\cref{egslope}中其他条件不变,但质点一开始有沿斜面向上的初速度$v_0 = 5\unit{m/s}$。
\end{eg}
\begin{ans}
(质点向上运动和下滑过程中摩擦力一样吗?)
\vspace{8cm}
\end{ans}


\subsection{多物体的多过程分析:物块-木板模型}
\begin{itemize}
\item
两物体画在同一个$v-t$图上 \so 两条线夹成的面积=相对位移(为什么?两个角度证实)
\item
$v-t$图判断共速时的时间和速度、共速前的位移 \so 是否会掉下木板
\end{itemize}
\begin{eg} \label{egcom}
如\cref{comoving},光滑水平面上有一个质量为$M = 2\unit{kg}$的无限长木板,另有一质量为$m = 1\unit{kg}$的木块以$v_0 = 6\unit{m/s}$的速度从左端冲上木板,它们随即开始向右运动。木板和木块之间的动摩擦因数为$0.4$。求两者的最终速度和木块在木板上划过的长度。
\end{eg}
\cpicn{0.6}{comoving}{物块-木板模型}
\begin{ans}
对物块应用牛顿第二定律:
\par
对木板应用牛顿第二定律:
\par
画出$v-t$图,两线的交点表明木板和物块达到相对静止,不再相对滑动。从图中读出此时的时间为$t_c = $\hspace{2cm},速度为$v_c =$\hspace{2cm}。围成的面积为相对位移,为$\Delta x =$\hspace{2cm}。
\vspace{4cm}
\end{ans}
\begin{itemize}
\item
如果水平面光滑,则有$mv_0 = (m+M)v_c$(计算验证一下) \so 动量守恒定律的体现

\item
如果有外力牵引或地面对木板有摩擦:共速后整体法分析得到加速度$a_c$,判断$a_c$和$\mu g$的相对大小 \so $a_c>\mu g$,不能一起运动,需要继续用隔离法分析。
\end{itemize}

\begin{eg}
\cref{egcom}中的其他条件不变,但
\begin{enumerate}
\item
光滑水平面改成$\mu = 0.1$的粗糙水平面;
\item
光滑水平面改成$\mu = 0.8$的粗糙水平面;
\item
水平面光滑,但有水平向右的恒力$F = 18\unit{N}$作用在木板上。木板长度有限,为$L = 6.6\unit m$。求物块在木板上运动的总时间
\end{enumerate}
求两者的最终速度和木块在木板上划过的长度;并求物块最终停在木板上的位置。
\end{eg}
\begin{ans}
(记得讨论最终能不能共同运动)
\vspace{8cm}
\end{ans}




\newpage
\section{能量和动量}
\subsection{功和势能}
\emph{初中的时候为什么引入功的概念?}
\begin{itemize}
\item
\info{功是能量转换的量度}:e.g. 电做了多少功=多少电能转换成了其他的能量。
\item
人把重物往上提:人消耗自己干饭获得的能量增加重物的势能 \so 生活经验:越重、举得越高,消耗能量越大 \so 转换的能量=功=$Fs$。
\item
力$\vec F$,位移$\vec s$都是矢量 \so $W = \vec F \cdot \vec s = Fs \cos \theta$。\so 垂直于物体运动方向不做功,于运动方向相反的力做负功。
\item
合力(矢量和)做功=各分力做功之和(代数和)
\item
重物向上提,重力势能增大。同时,重力向下,位移向上,重力做负功 \so 什么力做负功,什么力对应的能量就增大。
\item
滑动摩擦力方向和位移相反 \so 摩擦力做负功,热能(内能)增加。
\item
不是所有力都有势能:高中范围内,只有重力(引力)、弹力和电场力有相应的势能。
\end{itemize}
\emph{势能的表达式是什么?}
\begin{itemize}
\item
重力做了多少负功=重力势能增大了多少 \so 取地面为参考平面(高度$h = 0$),$E_g = -(-mg)\cdot h = mgh$。
\item
弹性势能的表达式是什么呢?力$F$随$x$变化...和$v-t$图下的面积为位移一样的思想 \so $E_e = \frac{1}{2}kx^2$。(不要求,从量纲分析的角度能知道$E\propto kx^2$)
\end{itemize}

\subsection{动能和动能定理}
\begin{itemize}
\item
公式$v^2_2 - v_1^2 = 2as$两边同乘$\frac{1}{2}m$ \so 
\beq
\frac{1}{2}mv_2^2 - \frac{1}{2}mv_1^2 = Fs
\eeq
\item
等式右边是功,暗示左边是某种能量 \so 定义动能$E_k = \frac{1}{2}mv^2$。
\item
动能定理:合外力所做的功等于物体动能的变化量。(对任意运动都成立)
\end{itemize}

\subsection{机械能守恒定律}
\begin{itemize}
\item
只有重力做功的动能定理:物体从高度$h_1$运动到$h_2$
\beq
\frac{1}{2}mv_2^2 - \frac{1}{2}mv_1^2 = (-mg)(h_2-h_1)
\eeq
\so
\beq
\frac{1}{2}mv_1^2 + mgh_1 = \frac{1}{2}mv_2^2  + mgh_2
\eeq
\item
物理意义:只有重力做功时,动能+势能等于常数
\item
推广到有弹力的情况和多个物体组成的系统\so 定义机械能=动能+势能 \so \info{只有重力和弹力做功(无摩擦力)时,物体(系统)的机械能守恒}。【机械能守恒定律】
\item
再推广:能量守恒定律
\end{itemize}
\emph{为什么初中的时候告诉我们使用机械只能省力不能省功?}

\subsection{动量定理和动量守恒}
\emph{由$v^2_2 - v_1^2 = 2as$可以得到动能定理,那由另一个相似的公式$v_2 - v_1 = a\Delta t$可以得到什么类似的结论?}
\begin{itemize}
\item
$v_2 - v_1 = a\Delta t$ 两边同乘质量$m$ \so $mv_2 - mv_1 = F \Delta t$ \so 定义动量$p = mv$,冲量$I = F\Delta t$
\item
动量定理:
\beq
I = \Delta p
\eeq
\item
是矢量等式!
\item
动量定理和动能定理都是牛顿第二定律的推论,不是独立的。
\end{itemize}
\begin{eg}
一个质量为$m = 0.1\unit{kg}$的小球以$v = 100\unit{m/s}$的速度垂直地砸向一个弹性墙壁,在$0.2\unit{s}$后被以原速度大小弹回,求弹性墙壁形变最大时的弹性势能和墙壁受到的平均冲力大小。
\end{eg}
\begin{ans}
形变最大时,小球不再深入墙壁,所有动能转换为弹性势能
\beq
E_{p,max} = \frac{1}{2}mv^2 = 500\unit J
\eeq
由动量定理,冲力为
\beq
F = \frac{\Delta p}{\Delta t} = \frac{mv - (-mv)}{\Delta t} = 100\unit N
\eeq
\end{ans}
\emph{机械能守恒的动量类比是什么}
\begin{itemize}
\item
两个物体$A,B$组成的系统,若系统不受外力而只存在内力:分别写出动量定理
\bea
\Delta \vec p_A &= \vec F_{BA} \Delta t \\
\Delta \vec p_B &= \vec F_{AB} \Delta t
\eea
牛顿第三定律保证$\vec F_{BA} +  \vec F_{AB} = 0$ \so $\Delta (\vec p_A + \vec p_B) = 0$ \so \info{不受合外力的系统动量不变}【动量守恒定律】
\item
动量守恒和机械能守恒的适用范围区别:动量守恒无外力(可以有内部的摩擦力);机械能守恒无摩擦力(可以有外部的重力)
\item
jargon:弹性碰撞=碰撞前后总机械能不变
\end{itemize}
\begin{eg}
如\cref{bullet}。一个质量为$M$的沙袋用绳悬挂在空中,一个质量为$m$的子弹以$V$的速度水平打入沙袋后和沙袋一起运动,求沙袋上升的最大高度。有多少能量转化成了热能? 
\end{eg}
\cpicn{0.6}{bullet}{子弹和沙袋}
\begin{ans}
子弹和沙袋组成的系统在子弹打入的瞬间不受合外力作用,满足动量守恒
\beq
mV = (m+M)v
\eeq
其中$v$是子弹打入后瞬间沙袋的速度。此后两者相对静止,无摩擦耗散,由机械能守恒由
\beq
\frac{1}{2}(m+M)v^2 = (m+M)gh
\eeq
联立两式容易解得
\beq
h = \frac{(mV)^2}{2(m+M)^2g} \approx \lbk\frac{m}{M} \rbk^2\frac{V^2}{2g}
\eeq
\end{ans}
\begin{itemize}
\item
这一题目的实际应用?测量$h$ \so 测量子弹速度!
\end{itemize}

\newpage
\section{直线运动综合问题}
\begin{itemize}
\item
原则上来说,需要用机械能解决的直线运动问题也可以全部用牛顿第二定律解决 \so 求加速度和$v-t$图是永远的重点和工具。
\item
2017年高考后引入了动量作为必考点 \so 有一些涉及碰撞的问题需要用动量守恒解决,并且速度可能突变(如果没有碰撞,速度不可能突变)。
\end{itemize}
\begin{eg} \label{15w}
(2015全国卷2)下暴雨时,有时会发生山体滑坡或泥石流等地质灾害。某地有一倾角$\theta = 37^\circ$的山坡$C$,上面有一质量为$m$的石板$B$,其上下表面于斜坡平行;$B$上有一碎石堆$A$(含有大量泥土),$A$和$B$均处于静止状态,如\cref{15gk}所示。假设某次暴雨中,$A$浸湿雨水后总质量也为$m$(可视为质量不变的滑块),在极短时间内,$A,B$间的动摩擦因数$\mu_1$减小为$\frac{3}{8}$,$B,C$间的动摩擦因数$\mu_2$减小为$0.5$,$A,B$开始运动,此时刻为计时起点;在第$2\unit s$末,$B$的上表面突然变为光滑,$\mu_2$保持不变。已知$A$开始运动时,$A$离$B$下边缘的距离$l = 27\unit m$,$C$足够长,设最大静摩擦力等于滑动摩擦力。取$g = 10\unit{m/s^2}$,$\sin \theta = 0.6$,求
\begin{enumerate}
\item
$0-2\unit s$时间内$A$和$B$的加速度大小。
\item
$A$在$B$上总的运动时间。
\end{enumerate}
\end{eg}
\cpicn{0.25}{15gk}{\cref{15w}题图}
\begin{ans}
\ 
\vspace{8cm}
\end{ans}

\begin{eg} \label{17w}
(2017全国卷3)如\cref{17gk},两个滑块$A$和$B$的质量分别为$m_A = 1\unit{kg}$和$m_B = 5\unit{kg}$,放在静止于水平地面上的木板的两端,两者与木板间的动摩擦因数为$\mu_1 = 0.5$;木板的质量为$m = 4\unit{kg}$,与地面间的动摩擦因数为$\mu_2 = 0.1$。某时刻$A,B$两滑块开始相对滑动,初速度大小均为$v_0 = 3\unit{m/s}$。$A,B$相遇时,$A$与木板恰好相对静止。设最大静摩擦力等于滑动摩擦力,取重力加速度大小$g=10\unit{m/s^2}$,求
\begin{enumerate}
\item
$B$与木板相对静止时,木板的速度;
\item
$A,B$开始运动时,两者之间的距离
\end{enumerate}
\end{eg}
\cpicn{0.32}{17gk}{\cref{17w}题图}
\begin{ans}
\ 
\vspace{8cm}
\end{ans}
\begin{itemize}
\item
关于弹性碰撞的简单结论:
\begin{itemize}
\item
弹性撞墙:速度反向(类比光的反射);
\item
两个质量相等的物体弹性碰撞:交换速度。
\end{itemize}
\item
其他情形:联立能量守恒(碰撞前后动能相等)和动量守恒求解。
\end{itemize}
\newpage
\begin{eg}
(2019全国卷3)静止在水平地面上的两小物块$A,B$,质量分别为$m_A = 1.0\unit{kg}, m_B = 4.0\unit{kg}$;两者之间有一被压缩的微型弹簧,$A$与其右侧的竖直墙壁距离$l = 1.0\unit{m}$,如图所示。某时刻,将压缩的微型弹簧释放,使$A,B$瞬间分离,两物块获得的动能之和为$E_k = 10.0\unit J$。释放后,$A$沿着与墙壁垂直的方向向右运动。$A,B$与地面之间的动摩擦因数均为$\mu = 0.20$。重力加速度取$g = 10 \unit{m/s^2}$。$A,B$运动过程中所涉及的碰撞均为弹性碰撞且碰撞时间极短。
\begin{enumerate}
\item
求弹簧释放后瞬间$A,B$速度的大小
\item
物块$A,B$中的哪一个先停止?该物块刚停止时$A$与$B$之间的距离是多少?
\item
$A,B$都停止后,$A$与$B$之间的距离是多少?
\end{enumerate}
\end{eg}
\begin{ans}
\ 
\vspace{8cm}
\end{ans}




\newpage
\section{平抛运动}
\subsection{速度合成}
\emph{如何描述平面内的运动?}
\begin{itemize}
\item
速度是矢量\so 加法遵循平行四边形定则。
\item
小船过河问题:速度合成,船在静水中的速度$\vec v_b$,河水速度$\vec v_r$ \so 实际速度$\vec v = \vec v_b + \vec v_r$。如\cref{boat}。
\cpicn{0.8}{boat}{小船过河的速度合成}
\item
以最短时间和最短路程过河。
\end{itemize}
\begin{eg}
河两岸之间的距离为$h$,船速大小为$v_b$,方向可以自由调节;水速为$v_r$平行与河岸,求小船过河的最短时间和最短路程。
\end{eg}
\begin{ans}
最短路程需要分类讨论:看合速度$\vec v$可能落在平面上的什么范围内。
\vspace{6cm}
\end{ans}

\subsection{曲线运动}
\emph{轨迹为曲线时,如何描述质点在某一点的速度方向?}
\begin{itemize}
\item
按定义
\beq
\vec v = \frac{\vec r(t + \Delta t) - \vec r(t)}{\Delta t}
\eeq
其中$\vec r(t)$是质点在$t$时刻的位置,如\cref{posvec}。
\cpicn{0.5}{posvec}{质点的位置可以用一个矢量$\vec r = (x,y)$描述,$x,y$都是$t$的函数}
\item
$\Delta t$越小,越接近切线 \so 瞬时速度沿曲线切线方向,如\cref{tvec}。
\cpicn{0.5}{tvec}{速度方向是曲线某点的切线方向}
\item
数轴图像\so 坐标系图像
\item
初中知识:力是改变速度的原因,$\vec F,\vec v$不共线则做曲线运动 \so 曲线运动需要力来维持,存在加速度;曲线运动速度方向在不断改变\so 曲线运动无论如何都是变速运动
\end{itemize}

\subsection{平抛运动}
\emph{为什么二次函数的图像被称为抛物线?}
\begin{itemize}
\item
如\cref{throwing},将质点从$h$的高度以$v_0$的初速度水平抛出
\begin{itemize}
\item
竖直方向受重力\so 自由落体运动 \so $y = - \frac{1}{2}gt^2 + h$;
\item
水平方向不受力 \so 匀速直线运动 \so $x = v_0 t$。
\end{itemize}
\item
两式消去$t$:
\beq
y = -\frac{g}{2v_0^2}x^2 + h
\eeq
轨迹为一个二次函数。
\cpicn{0.6}{throwing}{平抛运动的轨迹}
\item
数学上
\beq
\begin{cases}
x &= v_0 t,\\
y &= -\frac{1}{2}gt^2 + h
\end{cases}
\eeq
称为参数方程。物理意义:描述质点在$t$时刻的位置。
\item
用$y = -\frac{1}{2}gt^2 + h$还是$y = \frac{1}{2}gt^2$取决于具体问题中所建立的$y$轴方向。
\item
在$t$时刻的\info{速度}为$v_x = v_0,v_y = gt$,\info{速率}为$\sqrt{v_0^2 + (gt)^2}$。
\end{itemize}
\begin{eg}
轰炸机在高度为$h$的高空以$v$匀速飞行,欲摧毁一个地面目标,不计空气阻力,请问轰炸机需要在到达目标正上空前提前多少时间投弹?
\end{eg}
\begin{ans}
定性分析:由于惯性需要提前投弹,高度越高提前时间越长 \so 假设需要提前的时间为$t$,满足方程
\vspace{4cm}
\end{ans}
\begin{eg}
如\cref{tos},从倾角为$\alpha$的足够长斜面顶端以$v$的速度平抛一个质点,求其在斜面上的落点到斜面顶端的距离。
\end{eg}
\cpicn{0.6}{tos}{斜面平抛模型}
\begin{ans}
方法1:在顶端建立平面直角坐标系$xOy$,联立轨迹方程和直线方程...
\vspace{4cm}
\par
方法2:利用质点在$t$时刻的位置满足$\frac{y}{x} = \tan \alpha$,有方程...
\vspace{4cm}
\end{ans}
\par
\emph{如果斜面长度是有限的,怎么分类讨论?}


\newpage
\section{匀速圆周运动}
\subsection{角速度和线速度}
\emph{怎么严格定义匀速圆周运动?}
\begin{itemize}
\item
轨迹沿圆周(圆周),速率始终不变(匀速)。
\item
\info{线速度}为不变的速率:
\beq
v = \frac{\text{在$\Delta t$时间内走过的弧长}}{\Delta t} = \frac{\Delta s}{\Delta t}
\eeq
\item
\info{角速度}为单位时间内走过的圆心角弧度数:
\beq
\omega = \frac{\text{在$\Delta t$时间内转过的弧度数}}{\Delta t} = \frac{\Delta \theta}{\Delta t}
\eeq
\item
$\omega$的单位$\text{弧度}/{\rm s} ={\rm rad/s} ={\rm s^{-1}}$。
\item
曲线运动速度方向沿曲线切线方向 \so 匀速圆周运动速度方向沿该点切线方向。
\item
弧度的定义$\Delta \theta = \frac{\Delta s}{r}$ \so 
\beq
v = \omega r
\eeq
\item
一图总结:\cref{circle}。
\cpicn{0.5}{circle}{匀速圆周运动的速度和角速度}
\item
匀速圆周运动的周期=转一周所花费的时间。一周=$2\pi$ \so 
\beq
T = \frac{2\pi}{\omega}
\eeq
\end{itemize}
\emph{如果角的单位用度($^\circ$)而非弧度,$v = \omega r$还成立吗?}
\begin{eg}
如\cref{chilun},两个齿轮的半径分别为$r_1$和$r_2$。当左边齿轮以$n_1 \text{圈/s}$的转速转动时,右边齿轮的转速是多少?
\end{eg}
\cpicn{0.4}{chilun}{两个贴合的齿轮}
\begin{ans}
分析:两个齿轮如此贴合,连接处的线速度必须相等。
\vspace{3cm}
\end{ans}
\par
思考:\emph{汽车离合器的原理是什么?}
\subsection{向心加速度}
\emph{匀速圆周运动是变速运动,其加速度是多大?}
\begin{itemize}
\item
量纲分析的角度:$a$的单位为${\rm m/s^2}$,$a$可能取决于$\omega$和$r$ \so $a \propto \omega^2 r = \frac{v^2}{r}$
\item
运动学严格分析:几何学角度,相似三角形给出$\frac{\Delta s}{r} \approx \frac{\Delta l}{r} = \frac{\Delta v}{v}$ \so $\Delta v = \frac{v^2}{r} \Delta t$ \so 
\beq
a = \frac{v^2}{r} = \omega^2 r
\eeq
方向指向圆心。
\item
推导过程中没有用到牛顿定律 \so 向心加速度是$F=ma$的右边而非左边,$m\frac{v^2}{r}$不是力。
\item
向心加速度公式对非匀速圆周运动也适用,$v$是在某点的瞬时线速度(速率)。此时的向心加速度也称法向加速度$a_n$(方向指向法线方向)。
\end{itemize}
\begin{eg}
水平面内一个质点被长度为$l$的绳子牵引围绕原点$O$做匀速圆周运动,质点的动能为$E_k$,求绳子对小球的拉力为多少。
\end{eg}
\begin{ans}
小球做匀速圆周运动对应的向心加速度为$a_n = \frac{v^2}{r}$。由牛顿第二定律
\beq
F = ma_n = \frac{mv^2}{l}
\eeq
由根据动能的定义$E_k = \frac{1}{2}mv^2$,有
\beq
F = \frac{2E_k}{l}
\eeq
\end{ans}

\begin{eg}
如\cref{cirmove},光滑水平面上有一个小孔和一个质量为$m$的做匀速圆周运动的质点,其圆周运动的半径为$r$。一绳连接该质点穿过小孔,连接一个质量为$M$的重物。重物在竖直方向上静止,求质点的角速度。
\end{eg}
\cpicn{0.4}{cirmove}{平面穿绳模型}
\begin{ans}
分析:经验定性判断,如果质点角速度为0,肯定会被重物拉掉下平面;如果质点运动太快,重物肯定会被拉飞 \so 存在一个合适的速度保持平衡,应该是绳上拉力刚好提供向心加速度的时候(和上面例题一样)。
\par
设绳上的张力为$T$,对重物运用牛顿第二定律,有
\vspace{4cm}
\end{ans}

\begin{eg}
如\cref{semicir},半圆形轨道的半径为$R$,一个质量为$m$的质点以$v_0$的速度冲上轨道。
\begin{enumerate}
\item
如果质点不从最高点落下,$v_0$至少为多大?
\item
如果质点在最高点对轨道的压力为$N$,其离开轨道后在水平面上的落点距离轨道下端多远?
\end{enumerate}
\end{eg}
\cpicn{0.6}{semicir}{半圆轨道模型}
\begin{ans}
定性分析:如果速度较小,肯定会从最高点落下 \so 有一个合适的速度使得质点不会落下,在最高点的速度应该刚好满足重力提供向心加速度。
\par
从轨道最低点到最高点,设在最高点的速度为$v$,由机械能守恒定律(或动能定理):
\beq
\frac{1}{2}mv_0^2 = \frac{1}{2}mv^2 + mg(2R)
\eeq
在最高点,由牛顿第二定律:
\vspace{5cm}
\end{ans}
\subsection{离心现象}
\emph{洗衣机甩干衣服的原理是什么?}
\begin{itemize}
\item
匀速圆周运动需要时刻指向圆心的外力维持 \so 当外力突然不足(小于$\frac{mv^2}{r}$),物体远离圆心运动 \so 离心现象。
\item
反过来,如果指向圆心的外力突然增大 \so 靠近圆心运动。
\item
典型例子:洗衣机、印度飞饼...
\end{itemize}
\par
思考:\emph{一个自转的圆盘需要力维持旋转吗?什么力承担了这个角色?}


\newpage
\section{万有引力定律}
\subsection{万有引力}
\emph{为什么万有引力普遍存在于物体间,却被人类忽视了几千年?}
\begin{itemize}
\item
\info{据说}牛顿被苹果砸了一下,然后就想出了万有引力定律,如\cref{gravitation}。
\cpicn{0.8}{gravitation}{高空抛物的危害性}
\item
既然地球和苹果之间有,那任何两个物体之间都应该有 \so 物理定律有普适性,不偏爱任何一个特殊物体。
\item
质量分别为$m_1$和$m_2$的物体(质点),间距$r$为质心的间距,万有引力方向指向对方,大小为
\beq
F = G \frac{m_1 m_2}{r^2}
\eeq
其中$G = 6.67\times 10^{-11}\unit{N\cdot m^2/kg^2}$(化成基本单位是什么?)。
\item
为什么叫$G$? \so 引力=Gravitation
\item
为什么是$m_1m_2$? \so 牛顿第三定律的要求。
\item
为什么正比于$\frac{1}{r^2}$? \so 类比于一个灯泡在$r$距离的亮度(单位面积上的功率),如\cref{revsqr}。 \so 事实上因为我们生活在3维空间,所以电场力也是$\frac{1}{r^2}$的。
\cpicn{0.6}{revsqr}{为什么是平方反比}
\end{itemize}
\begin{eg}
根据万有引力定律,重力加速度$g$可以怎样用地球质量$M$和地球半径$R$,以及万有引力常数表示?怎么用地球密度$\rho$和$R$已经$G$表示?在距地球表面高度$h$的山上,重力加速度$g(h)$和地面重力加速度$g(h=0) = g$的关系是什么?
\end{eg}
\begin{ans}
在地球表面,质点受到的重力就是地球对质点的万有引力。于是
\vspace{6cm}
\end{ans}

\subsection{行星轨道和开普勒定律}
\emph{地理知识:太阳系内行星的轨道有什么特点?}
\begin{itemize}
\item
行星轨道具有近圆性,近似认为它们就是半径为$r$的圆,行星在轨道上做匀速圆周运动,角速度为$\omega$。太阳质量记为$M_\odot$。\so 牛顿第二定律,万有引力提供向心力
\beq
\frac{GM_\odot m}{r^2} = m\omega^2 r
\eeq
\item
天文上更常用的是周期$T$(大家都知道地球绕太阳转的周期是1年,但是很少说角速度是多少),$\omega = \frac{2\pi}{T}$ \so
\beq
\frac{GM_\odot m}{r^2} = m\lbk \frac{2\pi}{T}\rbk^2 r
\eeq
\so
\beq \label{k3}
\frac{r^3}{T^2} = \frac{GM_\odot}{4\pi^2}
\eeq
右边对太阳系的所有行星都是一个常数。(开普勒第三定律)
\item
历史上,开普勒\info{先从天文观测数据}总结出\cref{k3},此外还有开普勒第一和第二定律:
\begin{enumerate}
\item
(第一定律)行星的轨道都是一个椭圆,太阳在椭圆的一个焦点上。
\item
(第二定律)行星与太阳的连线在相同时间扫过相同的面积。
\item
(第三定律)行星椭圆轨道半长轴$a$的三次方和周期$T$的平方成正比。
\end{enumerate}
\item
初学第一次不用理解椭圆、焦点这些概念,这也不是考试重点。圆是椭圆的退化情况,此时椭圆的焦点退化为圆心;半长轴退化为半径。
\item
太阳质量$M_\odot = 2.0\times 10^{30}\unit{kg}$,如何从天文观测数据上得到?怎么得到地球的质量?
\end{itemize}
\begin{eg}
地球和太阳的平均距离$r_\oplus$常记做一个天文单位$r_\oplus = 1\unit{AU}$。若某一个太阳系行星到太阳的距离是$a\unit{AU}$,其周期是多少年?
\end{eg}
\begin{ans}
根据开普勒第三定律,有
\vspace{3cm}
\end{ans}
\begin{eg}
绕地球以半径为$r$的圆轨道运行的卫星的速度为多大?卫星越贴近地球表面,卫星的速度是增大还是减小?卫星的最大可能速度称为第一宇宙速度$v_1$,求第一宇宙速度的表达式,分别用$G,M,R$和$g,R$表示。
\end{eg}
\begin{ans}
由牛顿第二定律,万有引力提供向心加速度,有
\vspace{6cm}
\end{ans}


















\end{document}